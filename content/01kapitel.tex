%!TEX root = ../dokumentation.tex

\chapter{Aufgabe}
Erste Erwähnung eines Akronyms wird als Fußnote angezeigt. Jede weitere wird
nur verlinkt: \acf{AGPL}. \cite{fsf:2007}

Verweise auf das Glossar: \gls{Glossareintrag}, \glspl{Glossareintrag}

Nur erwähnte Literaturverweise werden auch im Literaturverzeichnis gedruckt:
\cite{baumgartner:2002}, \cite{dreyfus:1980}

Meine erste Fußnote\footnote{Ich bin eine Fußnote}

\begin{wrapfigure}{r}{.4\textwidth}
\centering
\includegraphics[height=.35\textwidth]{logo.png}
\vspace{-15pt}
\caption{Das Logo der Musterfirma\footnotemark}
\end{wrapfigure}
%Quelle muss in Fußnote stehen (da sonst aufgrund eines Fehlers nicht kompiliert
% wird)
\footnotetext{aus \cite{mustermann:2012}}

\section{Carolocup, Regeln}
\section{Simulation}
\section{Anforderungen}

Immer wenn ich Fahrzeug sage meine ich das simulierte, wenn ich echtes Fahrzeug sage meine ich das echte.
