\chapter{Umfeld}

\section{ROS}
Das \acf{ROS} ist ein Betriebssystem das speziell auf Roboter angepasst ist.
Es läuft auf einem anderen Betriebssystem, in diesem Fall Ubuntu.
Es stellt Funktionalitäten wie Hardware Abstraktion, Paket Management, Kommunikation zwischen Prozessen... zu verfügung.
Das Ziel von \acf{ROS} ist die Wiederverwendbarkeit von Code. 
Dies gelingt dadurch das \acf{ROS} ein verteiltes System an Prozessen (Nodes) ist. 
Die Nodes kommunizieren untereinander über Topics.
Topics sind die Busse auf denen Nodes Nachrichten senden.
Nachrichten zu senden oder zu empfangen ist anonym, dadurch sind Sender und Empfänger komplett entkoppelt voneinander.
\cite{rosTopics:2019}
Nodes lassen sich so austauschen und wiederverwenden.
\cite{rosIntro:2018}
Die Regelung des Fahrzeugs läuft auf einer ROS Architektur.


\section{Gazebo}
Gazebo ist ein open-source Robotersimulator.
Er beinhalted eine 3D-Umgebung mit Physik-Engine, Sensorsimulation und Antriebssimulation.
Gazebo verfügt außerdem über ein graphisches Interface.
Ein Vorteil von Gazebo ist die aktive Community und die vielen Plugins.
\cite{Gazebo}

\section{KIT Gazebo Simulation}
Als Ausgangssituation für die Simulation wird die open source Simulationsumgebung des KITs genutzt.
Diese ist eine Gazebo Simulation mit verschiedenen ROS Nodes.
Die wichtigsten Nodes sind folgende:
\subsection*{CarStateNode}
Publisht sämtliche Informationen über den aktuellen Zusand des Fahrzeugs. 
Dazu gehören die Position, die Orientierung, die Geschwindigkeit und ein Field of View.
\cite{KITcar:gazeboSimulationPackage}
\subsection*{GazeboRateControlNode}
Diese Node ist nur nötig wenn die Simulation in einem Docker Container läuft.
Gazebo gewärleistet keine update-Rate innerhalb eines Docker Conteiners.
In dem Fall kann sich die GazeboRateControlNode darum kümmern.
\cite{KITcar:gazeboSimulationPackage}
\subsection*{ModelPluginLinkNode}
Erlaubt eine leichte Interaktion mit Gazebo Modellen, über bereitgestellte Topics.
\cite{KITcar:gazeboSimulationPackage}
\subsection*{WorldPluginLinkNode}
Erlaubt Hinzufügen und Entfernen von Gazebo Modellen, über bereitgestellte Topics.
\cite{KITcar:gazeboSimulationPackage}
\subsection*{AutomaticDriveNode}
Erlaubt es die Simulation ohne Fahrzeugsteuerung laufen zu lassen. 
Das Fahrzeug beweg sich entlang der idealen Fahrspur. 
\cite{KITcar:gazeboSimulationPackage}
\subsection*{Verschiedene Evaluierungs Nodes}
Das Fahrzeug wird evaluiert indem über verschiedene State Machines das Verhalten aufgezeichnet wird.
Außerdem werden Geschwindigkeit, Kollisionen und Spurhaltung aufgezeichnet.
\cite{KITcar:simulation_evaluation}
\subsection*{Groundthrouth Nodes}
Stellt Informationen über die aktuelle Straße zur Verfügung.
\cite{KITcar:simulation_groundtruth}
\subsection*{URDF Model}
Das URDF Model definiert das Aussehen des Fahrzeugs.
Auch Sensoren und Kamera sind im URDF Model definiert und angepasst.
Das Model ist jedoch statisch. 
Man übergibt dem Model direkt eine Transformation um es zu bewegen.
Dadurch spielen im URDF Model definierte Größen, wie das Gewicht, die Reibung oder Interaktionen zwischen Bausteinen innerhalb des Models, keine Rolle.
Bisher werden diese Größen vernachlässigt da man eine externe Regelung benötigt um die Simulation zu nutzen.
\cite{KITcar:gazeboSimulationPackage}

\section{Fahrzeugsteuerung}

Die eigentliche Fahrsimulation des Fahrzeugs ist nicht Teil der Simulationsumgebung.
Um das Fahrzeug zu bewegen erwarted die Simulation ein Topic auf dem neue Position gepublished werden.
Da die Simulation aber den Lenkwinkel als Input nutzen soll, muss die Simulation dahingehend erweitert werden.



Immer wenn ich Fahrzeug sage meine ich das simulierte, wenn ich echtes Fahrzeug sage meine ich das echte.