%!TEX root = ../dokumentation.tex

\chapter{Umfeld}

%title wird unter dem Bsp. abgedruckt
%caption wird im Verzeichnis abgedruckt
%label wird zum referenzieren benutzt, muss einzigartig sein.

% \begin{lstlisting}[caption=Code-Beispiel, label=Bsp.1]
% public class HelloWorld {
% 	public static void main (String[] args) {
% 		// Ausgabe Hello World!
% 		System.out.println("Hello World!");
% 	}
% }
% \end{lstlisting}

% %language ändert die Sprache. (Wenn nur eine Sprache verwendet wird, kann diese Sprache in einstellungen.tex geändert werden. Standardmäßig Java.)
% \begin{lstlisting}[caption=Python-Code, label=Python-Code, title=Titel des Python-Codes,language=Python]
% def quicksort(liste):
% if len(liste) <= 1:
% 	return liste
% pivotelement = liste.pop()
% links = [element for element in liste if element < pivotelement]
% rechts = [element for element in liste if element >= pivotelement]
% return quicksort(links) + [pivotelement] + quicksort(rechts)
% # Quelle: http://de.wikipedia.org/wiki/Python_(Programmiersprache)
% \end{lstlisting}

% \section{lorem ipsum}
% Looking for the one superhero comic you just have to read. Following the antics and adventures of May Mayday Parker, this Spider-book has everything you could want in a comic--action, laughs, mystery and someone in a Spidey suit. Collects Alias \#1-28, What If. Jessica Jones had Joined the Avengers. In her inaugural arc, Jessicas life immediately becomes expendable when she uncovers the potentially explosive secret of one heros true identity. 

% Once upon a time, Jessica Jones was a costumed super-hero, just not a very good one. First, a story where Wolverine and Hulk come together, and then Captain America and Cable meet up. In a city of Marvels, Jessica Jones never found her niche. The classic adventures of Spider-Man from the early days up until the 90s. Looking for the one superhero comic you just have to read.

% Meet all of Spideys deadly enemies, from the Green Goblin and Doctor Octopus to Venom and Carnage, plus see Peter Parker fall in love, face tragedy and triumph, and learn that with great power comes great responsibility. In a city of Marvels, Jessica Jones never found her niche. Bitten by a radioactive spider, high school student Peter Parker gained the speed, strength and powers of a spider. Looking for the one superhero comic you just have to read. What do you get when you ask the question, What if Spider-Man had a daughter.

% The classic adventures of Spider-Man from the early days up until the 90s. Amazing Spider-Man is the cornerstone of the Marvel Universe. But will each partner’s combined strength be enough. Adopting the name Spider-Man, Peter hoped to start a career using his new abilities. Youve found it.

% \section{Code}
% Verweis auf den Code \autoref{Bsp.1}.\\
% und der Python-Code \autoref{Python-Code}.

% Zweite Erwähnung einer Abkürzung \ac{AGPL} (Erlärung wird nicht mehr angezeigt)

\section{ROS}
Das \acf{ROS} ist ein Betriebssystem das speziell auf Roboter angepasst ist.
Es läuft auf einem anderen Betriebssystem, in diesem Fall Ubuntu.
Es stellt Funktionalitäten wie Hardware Abstraktion, Paket Management, Kommunikation zwischen Prozessen... zu verfügung.
Das Ziel von \acf{ROS} ist die Wiederverwendbarkeit von Code. 
Dies gelingt dadurch das \acf{ROS} ein verteiltes System an Prozessen (Nodes) ist. 
Die Nodes kommunizieren untereinander über Topics.
Topics sind die Busse auf denen Nodes Nachrichten senden.
Nachrichten zu senden oder zu empfangen ist anonym, dadurch sind Sender und Empfänger komplett entkoppelt voneinander.
\cite{rosTopics:2019}
Nodes lassen sich so austauschen und wiederverwenden.
\cite{rosIntro:2018}
Die Regelung des Fahrzeugs läuft auf einer ROS Architektur.


\section{Architektur}
Als Ausgangssituation für die Simulation wird die open source Simulationsumgebung des KITs genutzt.
Diese ist eine Gazebo Simulation mit verschiedenen ROS Nodes.
HIER ARCHITEKTUR BILD EINFÜGEN
Die eigentliche Fahrsimulation des Fahrzeugs ist nicht Teil der Simulationsumgebung.
Um das Fahrzeug zu bewegen erwarted die Simulation ein Topic auf dem neue Position gepublished werden.
Da die Simulation aber den Lenkwinkel als Input nutzen soll, muss die Simulation dahingehend erweitert werden.


\section{Hardware}
